\documentclass{article}
\usepackage{amsmath}
\usepackage{amssymb}
\usepackage{amsthm}
\usepackage{fixdif}
\usepackage{enumitem}

\counterwithin{equation}{section}
\newtheorem*{lemma}{Lemma}
\newtheoremstyle{remarks}
                {3pt} % 上方间距
                {3pt} % 下方间距
                {\rmfamily} % 主体字体
                {\parindent} % 缩进长度
                {\itshape} % 定理头字体
                {.} % 标点符号
                { } % 定理头后间距
                {} % 定理头格式指定
\theoremstyle{remarks}
\newtheorem*{remarks}{Remarks}

\renewcommand{\div}{\operatorname{div}}
\title{An Optimal Poincar\'e Inequality for Convex Domains}
\author{L. E. Payne \and H. F. Weinberger}
\date{1960}
\begin{document}

\maketitle

\section{Introduction}

Let $G$ be a convex $n$-dimensional domain with boundary $C$.
It is easily seen that the lowest eigenvalue of the free membrane problem
\begin{equation}\label{eq:1.1}
  \begin{aligned}
    \Delta v + \nu v & = 0 \quad \text{in}\ G \\
    \partial v/\partial n & = 0 \quad \text{on}\ C 
  \end{aligned}
\end{equation}
is zero, the eigenvalue beging any constant.

This corresponds to the fact that the solution of the interior
Neumann problem
\begin{equation}\label{eq:1.2}
  \begin{aligned}
    & \Delta \varphi = 0 \quad\quad \text{in}\ G \\
    & \partial\varphi/\partial n \quad \text{given on}\ C
  \end{aligned}
\end{equation}
is only determined to within a constant. The latter is to be fixed by a normalization
such as
\begin{equation}\label{eq:1.3}
  \int_G \varphi \d G = 0. 
\end{equation}

The authers have previously introduced a method for bounding the pointwise
value as well as the Dirichlet integral of a solution $\varphi$ of the exterior
Neumann in terms of a boundary integral of $(\partial\varphi/\partial n)^2$ [3].
In order to extend this method to the interior Neumann problem one needs
a lower bound for the second eigenvalue $\mu_2$ of~\eqref{eq:1.1}.
This eigenvalue is characterized by the minimum principle
\begin{equation}\label{eq:1.4}
  \mu_2 = \min_{\int_G u\d G=0} \frac{\int_G |\nabla u|^2 \d G}{\int_G u^2 \d G}.
\end{equation}
A lower bound for $\mu_2$ can be used in the interior Neumann problem
in the following manner (cf.~[3]).
Let $\vec{f}$ be a vector field which is piecewise continuously differentiable
throughout $G$ and points outward on $C$, so that
\begin{equation}\label{eq:1.5}
  \vec{f}\cdot\vec{n} \geq k > 0 \quad \text{on}\ C.
\end{equation}
For example, of $G$ is star-shaped with respect to the origin, we may take $\vec{f}$
to be the radius vector. By the divergence theorem and the inequality $a^2+b^2\geq 2ab$
we have, if $\varphi$ is normalized by~\eqref{eq:1.3},
\begin{equation}\label{eq:1.6}
  \begin{aligned}
    \oint_C \varphi^2 \vec{f}\cdot\vec{n} \d C
    & = \int_G \bigl[\varphi^2\div\vec{f} + 2\varphi \vec{f}\cdot\nabla\varphi\bigr] \d G \\
    & \leq \int_G \varphi^2 \bigl[\div\vec{f} + |\vec{f}|^2\bigr] \d G
        + \int_G |\nabla\varphi|^2 \d G \\
    & \leq \bigl[1+\mu_2^{-1}\max\bigl(\div\vec{f} + |\vec{f}|^2\bigr)\bigr]
        \int_G |\nabla\varphi|^2 \d G \\
    & = \bigl[1+\mu_2^{-1}\max\bigl(\div\vec{f}+|\vec{f}|^2\bigr)\bigr]
        \oint_C \varphi \partial\varphi/\partial n \d C.
  \end{aligned}
\end{equation}
Consequently by Schward's inequality
\begin{equation}\label{eq:1.7}
  \oint_C \varphi^2 \vec{f}\cdot\vec{n} \d C
  \leq \bigl[1 + \mu_2^{-1} \max\bigl(\div\vec{f} + |\vec{f}|^2\bigr)\bigr]^2
  \oint_C \bigl(\vec{f}\cdot\vec{n}\bigr)^{-1} (\partial\varphi/\partial n)^2 \d C.
\end{equation}

If $\varGamma$ is a fundamental solution of Laplace's equation with its singularity
at the interior point $P$, we have
\begin{equation}\label{eq:1.8}
  \begin{aligned}
  |\varphi(P)|
    & = \biggl|\oint_C (\varGamma \partial\varphi/\partial n - \varphi \partial\varGamma/\partial n)
        \d C \biggr| \\
    & \leq \biggl\{ \oint_C \bigl(\vec{f}\cdot\vec{n}\bigr)^{-1}
        (\partial\varphi/\partial n)^2 \d C \biggr\}^{\frac12}
        \biggl(\biggl\{\oint_C f\cdot n \varGamma^2 \d C\biggr\}^{\frac12} \\
    & \quad + \bigl[1+\mu_2^{-1}\max\bigl(\div\vec{f}+|\vec{f}|^2\bigr)\bigr]
        \biggl\{\oint_C \bigl(\vec{f}\cdot\vec{n}\bigr)^{-1}
          (\partial\varGamma/\partial n)^2 \d C\biggr\}^{\frac12} \biggr).
  \end{aligned}
\end{equation}
Thus, if a lower bound for $\mu_2$ is known, $\varphi(P)$ may be explicitly
bounded in terms of the square integral of $\partial\varphi/\partial n$.

These results can be extended to general second order differential equations (cf.~[3]).

In this paper we shall show that for a convex domain $G$ in any number of
dimensions
\begin{equation}\label{eq:1.9}
  \mu_2 \geq \pi^2D^{-2}
\end{equation}
where $D$ is the diameter of $G$.
This is the best bound that can be given in terms of the diameter alone in
the sense that $\mu_2 D^2$ tends to $\pi^2$ for a parallepiped all
but one of whose dimensions shrink to zero.

The inequality~\eqref{eq:1.9} is in general false for non-convex domains.
In fact, for a sequence of domains which tends to two disjoint subdomains,
$\mu_2$ tends to zero. For the special class of domains $G$ which are
symmetric about all the coordinate planes of a rectangular coordinate
system and have the property that the intersection of $G$ with any line in a coordinate
direction is simply connected, the authors have previously obtained
an inequality of the form~\eqref{eq:1.9} with~$D$ replaced by the maximum
length of intersection of $G$ with a line in any coordinate direction [4].

A simple upper bound for $\mu_2$ for any $n$-dimensional domain $G$
in terms of its volume $V$ is given by the isoperimetric inequality
\begin{equation}\label{eq:1.10}
  \mu_2 \leq p_n^2 K_n^{\frac{2}{n}} V^{-\frac{2}{n}},
\end{equation}
where $K_n$ is the volume of the unit $n$-sphere and $p_n$
is the lowest positive root of the equation
\begin{equation}\label{eq:1.11}
  p J_{\frac12 n}'(p) - \biggl(\frac12 n - 1\biggr) J_{\frac12 n}(p) = 0.
\end{equation}
Equality is attained when $G$ is a sphere.

For $n=2$ this inequality was conjectured by Kornhauser \& Stakgold [2]
and proved by Szeg\"o [5]. For general $n$ the proof was given by one of
the authors [6].

The eigenvalue $\mu_2$ itself is of interest in a variety of problems arising
in mathematical physics.
In two dimensions it is proportional to the square of the cutoff frequency
of the lowest $H$-mode of a wave guide [2].
In three dimensions it is proportional to the lowest resonant frequency
of an acoustic resonator with perfectly rigid walls.
It is also inversely proportional to the relaxation time for diffusion
in a body with perfectly reflecting boundary.

The proof of the lower bound \eqref{eq:1.9} is based upon a lemma
concerning a class of Sturm-Liouville systems.
This lemma, which is of some interest in itself, is stated and proved in \S2.
The inequality~\eqref{eq:1.9} is proved for two dimensions in \S3 and for
higher dimensions in \S4.


\section{A one-dimensional lemma}

In order to prove the lower bound~\eqref{eq:1.9} we require a somewhat stronger
version of its one-dimensional analogue. It is the following lemma.

\begin{lemma}
  Let $p(y)$ be a non-negative convex function of $y$ defined on
  the interval $0\leq y\leq L$; then for any piecewise continuously
  differentiable fucntion $u(y)$ which satisfies
  \begin{equation}\label{eq:2.1}
    \int_0^L p(y)u(y) \d y = 0
  \end{equation}
  it follows that
  \begin{equation}\label{eq:2.2}
    \int_0^L p(y) [u'(y)]^2 \d y \geq \pi^2L^{-2} \int_0^L p(y) [u(y)]^2 \d y.
  \end{equation}
\end{lemma}

\begin{proof}
  We assume for the moment that $p$ is strictly positive and twice differentiable.
  Then the function $v$ which minimizes the quotient
  \begin{equation}\label{eq:2.3}
    \int_0^L pu'^2 \d y \biggm/ \int_0^L pu^2 \d y
  \end{equation}
  among functions $u$ satisfying~\eqref{eq:2.1} must satisfy the Sturm-Liouville system
  [1, p.~348]
  \begin{equation}\label{eq:2.4}
    \begin{aligned}
      & [pv']' + \lambda pv = 0, \\
      & v'(0) = v'(L) = 0,
    \end{aligned}
  \end{equation}
  where $\lambda$ is the minimum value of the quotient~\eqref{eq:2.3}.
  We divide the equation~\eqref{eq:2.4} by $p$, differentiate with respect
  to $y$, and introduce the new variable
  \begin{equation}\label{eq:2.5}
    w = v' p^{\frac12}.
  \end{equation}
  The function $w$ satisfies the Sturm-Liouville system
  \begin{equation}\label{eq:2.6}
    \begin{aligned}
      & w'' + \biggl[\frac12 \frac{p''}{p} - \frac34\frac{p'^2}{p^2}\biggr]w + \lambda w = 0, \\
      & w(0) = w(L) = 0.
    \end{aligned}
  \end{equation}
  Because of the convexity of $p$ the term in square brackets is non-positive.
  Hence, multiplying~\eqref{eq:2.6} by $w$ and integrating by parts, we obtain
  \begin{equation}\label{eq:2.7}
    \lambda \geq \frac{\int_0^L w'^2 \d y}{\int_0^L w^2 \d y}.
  \end{equation}
  Since $w(0)=w(L)=0$ the quotient on the right of~\eqref{eq:2.7} is bounded
  below by the first eigenvalue of the vibrating string with fixed ends. Thus
  \begin{equation}\label{eq:2.8}
    \lambda \geq \pi^2 L^{-2}.
  \end{equation}
  Since $\lambda$ is the minimum of the quotient~\eqref{eq:2.3},
  \eqref{eq:2.2} is proved when $p$ is strictly positive and twice differentiable.

  If $\tilde{u}$ is any function defined on the interval $0\leq y\leq L$,
  the function
  \begin{equation}\label{eq:2.9}
    u(y) = \tilde{u}(y) - \biggl[\int_0^L p \d y\biggr]^{-1} \int_0^L p\tilde{u} \d y
  \end{equation}
  will satisfy~\eqref{eq:2.1}. Hence~\eqref{eq:2.2} implies
  \begin{equation}\label{eq:2.10}
    \int_0^L p\tilde{u}'^2 \d y \geq \pi^2 L^{-2} 
    \biggl\{\int_0^L p\tilde{u}^2 \d y - \biggl[\int_0^L p \d y\biggr]^{-1}
      \biggl[\int_0^L p\tilde{u} \d y\biggr]^2\biggr\}.
  \end{equation}
  Clearly~\eqref{eq:2.10} is valid for the uniform limit of admissible
  functions $p$. In particular, then, $p$ may be any non-negative convex function of $y$.
  Thus the lemma is proved.
\end{proof}

\begin{remarks}
  \begin{enumerate}[label = \arabic*.]
    \item The convexity of $p$ was used only to show that the square bracket
      in~\eqref{eq:2.6} is non-negative. For this purpose it is sufficient
      to assume that $p^{-\frac12}$ is a concave function of $y$.
      Therefore the lemma actually holds under this weaker condition.
    \item By the minimax theorem [1, p.~352] we can show that if $p^{-\frac12}$
      is a concave function of $y$, the eigenvalues of the Sturm-Liouville
      system~\eqref{eq:2.4} satisfy the inequality
      \begin{equation}\label{eq:2.11}
        \lambda_k \geq (k-1)^2 \pi^2 L^{-2},\qquad k=1,2,\ldots.
      \end{equation}
    \item Equality in \eqref{eq:2.11} is obtained if and only if $p^{-\frac12}$
      is linear in $y$. If $p$ is assumed convex, it must then be constant.
  \end{enumerate}
\end{remarks}


\section{The two-dimensional case}

Let $G$ be a convex plane domain with boundary $C$.
Let $\mu_2$ be defined as the infimum\footnote{If the boundary
$C$ is smooth so that the problem \eqref{eq:1.1} possesses eigenvalues,
$\mu_2$ is the second eigenvalue of~\eqref{eq:1.1}.}
of the quotient
\begin{equation}\label{eq:3.1}
  \int_G |\nabla u|^2 \d G \biggm/ \int_G u^2 \d G
\end{equation}
among functions which have bounded second derivatives in $G$ and satisfy
\begin{equation}\label{eq:3.2}
  \int_G u \d G = 0.
\end{equation}

Let $u$ be such a function. Consider the set of lines through the centroid
of $G$. It follows from continuity that at least one such line divides $G$
into two convex subdomains of equal area over each of which the integral of~$u$
vanishes. We now divide each of these subdomains into two more convex subdomains
of equal area over each of which the integral of $u$ vanishes.

Continuing this process, we arrive after a finite number of steps at a division
of $G$ into convex subdomains $G_\nu$ of arbitrary small equal areas $A_\nu$.
Furthermore,
\begin{equation}\label{eq:3.3}
  \int_{G_\nu} u \d G = 0
\end{equation}
on each $G_\nu$.

Let $\varrho_\nu$ be the radius of the largest circle contained in $G_\nu$.
Then clearly
\begin{equation}\label{eq:3.4}
  A_\nu \geq \pi \varrho_\nu^2.
\end{equation}
Hence, if $A_\nu$ is sufficiently small, the width $\varrho_\nu$ of $G$
is less than a preassigned $\varepsilon$:
\begin{equation}\label{eq:3.5}
  \varrho_\nu \leq \varepsilon.
\end{equation}
This means that $G_\nu$ is contained between two parallel lines at distance $\varepsilon$.
We introduce a rectangular coordinate system with $x_2$-axis along one of these lines
and the $x_1$-axis tangent to one end of $G_\nu$.
Let $L_\nu$ be the length of the projection of $G_\nu$ on the $x_2$-axis.
Clearly, $L_\nu\leq D$. Let $p(y)$ be the length of the intersection
of $G_\nu$ with the line $x_2=y$. Then $p(y)\leq\varepsilon$.
Because of the convexity of $G_\nu$, $p(y)$ is convex in $y$.

Let $M$ be a bound for the absolute values of $u$ and its first and second
derivatives. Then by the mean value theorem
\begin{equation}\label{eq:3.6}
  \biggl|\int_{G_\nu} \biggl(\frac{\partial u}{\partial x_2}\biggr)^2 \d G
  - \int_0^{L_\nu} p(y) [u(0,y)']^2 \d y \biggr|
  \leq 2M^2 A_{\nu} \varepsilon,
\end{equation}
\begin{equation}\label{eq:3.7}
  \biggl|\int_{G_\nu} u^2 \d G - \int_0^{L_\nu} p(y) [u(0,y)]^2 \d y\biggr|
  \leq 2M^2 A_\nu \varepsilon,
\end{equation}
and
\begin{equation}\label{eq:3.8}
  \biggl|\int_{G_\nu} u \d G - \int_0^{L_\nu} p(y) u(0,y) \d y\biggr|
  \leq M A_\nu \varepsilon.
\end{equation}

Applying the inequality~\eqref{eq:2.10} of the lemma, we find, using~\eqref{eq:3.3}
and $L_\nu\leq D$, that
\begin{equation}\label{eq:3.9}
  \begin{aligned}
    \int_{G_\nu} |\nabla u|^2 \d G
    & \geq \int_{G_\nu} \biggl(\frac{\partial u}{\partial x_2}\biggr)^2 \d G \\
    & \geq \pi^2 D^{-2} \int_{G_\nu} u^2 \d G
        - 2M^2 \bigl(1+\pi^2 D^{-2} (1+\frac12 \varepsilon)\bigr) A_\nu \varepsilon.
  \end{aligned}
\end{equation}

We sum these inequalities over all the subdomains $G_\nu$.
The sum of the $A_\nu$ is the area of $G$. Since $\varepsilon$
is arbitrarily small, we obtain the inequality
\begin{equation}\label{eq:3.10}
  \int_G |\nabla u|^2 \d G \geq \pi^2 D^{-2} \int_G u^2 \d G.
\end{equation}
Since $u$ is any function with bounded second derivatives satisfying~\eqref{eq:3.2},
we have, by definition,
\begin{equation}\label{eq:3.11}
  \mu_2 \geq \pi^2 D^{-2}.
\end{equation}


\section{The $n$-dimensional case}

Let $G$ be a convex $n$-dimensional domain with boundary $C$ ($n\geq 3$).
We again define $\mu_2$ as the infimum of the Rayleigh quotient~\eqref{eq:3.1}
among fucntions $u$ having bounded second derivatives in $G$ and satisfying
the conditions~\eqref{eq:3.2}\footnote{If the boundary $C$ is smooth so that
the problem~\eqref{eq:1.1} possesses eigenvalues, $\mu_2$ is the second
eigenvalue of~\eqref{eq:1.1}.}.

Let $u$ be such a function. We consider the set of $(n-1)$-planes of the form
$ax_{n-1}+bx_n = c$ passing through the centroid of $G$. By continuity we find
that at least one of these planes divides $G$ into two subdomains of equal
$n$-volumes over each of which the integral of $u$ vanishes.
We divide these subdomains in the same way and continue the process until
$G$ is divided into subdomains $G_\nu$ of arbitrarily small $n$-volume $V_\nu^{[n]}$.
If $\varrho_\nu$ is the radius of the largest inscribed $n$-sphere, we have
\begin{equation}\label{eq:4.1}
  V_\nu^{[n]} \geq K_n \varrho_\nu^n
\end{equation}
where $K_n$ is the volume of the unit $n$-sphere. Hence, by a sufficiently
large number of subdivisions, we can make $\varrho_\nu$ less than a preassigned
$\varepsilon$. In a particular $G_\nu$ we introduce new rectangular coordinates
with the $x_1$-axis normal to these planes. We proceed to subdivide $G_\nu$
by means of planes of the form $ax_{n-1}+bx_n = c$ into subdomains on each of
which the integral of $u$ vanishes. We make these dividing planes pass through
the centroid of the projections on $x_1=0$ of the domains being divided.
After a finite number of such divisions we obtain subdomains $G_\nu'$
whise projections on $x_1=0$ have arbitrarily small $n-1$ volumes $V_\nu^{[n-1]}$.
It follows as before that if $V_\nu^{[n-1]}$ is sufficiently small,
the projection of $G_\nu'$ on $x_1=0$ lies between two parallel $(n-2)$-planes
at distance at most $\varepsilon$. We keep the $x_1$-direction fixed and choose
the $x_2$-direction of a new rectangular coordinate system in $G_\nu'$
perpendicular to these planes.

If $n>3$ we divide each $G_\nu'$ further by means of planes $ax_{n-1}+bx_n = c$
passing through the centroid of the projections on $x_1=x_2=0$
of $G_\nu'$ and of the succeeding domains.

In this way we eventually obtain a subdivision of $G$ into a finite number
of convex subdomains $G_\nu''$ over each of which the integral of $u$ vanishes.
Furthermore, each $G_\nu''$ is contained in a parallelepiped of the form
\begin{equation}\label{eq:4.2}
  \begin{aligned}
    & 0\leq x_i \leq\varepsilon, \qquad i=1,2,\ldots,n-1 \\
    & 0\leq x_n \leq L_\nu
  \end{aligned}
\end{equation}
with respect to suitable rectangular coordinates.

Let $p(y)$ be the $n-1$ volume of the intersection of $G_\nu''$
with $x_n=y$. Then $p(y)$ is convex because of the convexity of $G_\nu''$,
and $p(y)\leq \varepsilon^{n-1}$ by~\eqref{eq:4.2}.

The inequality
\begin{equation}\label{eq:4.3}
  \mu_2 \geq \pi^2 D^{-2}
\end{equation}
is now derived exactly as in \S3.

{\small This research was supported in part by the United States Air Force
through the Air Force Office of Scientific Research of the Air Research
and Development Command under Contract No.~49(638)-228.}


\begin{thebibliography}{6}
  \bibitem{1} Courant, R., \& D. Hilbert: Methoden der Mathematischen Physik,
    vol.~1. Berlin: Springer 1931.
  \bibitem{2} Kornhauser, E. T., \& I. Stakgold: A Variational Theorem for
    $\nabla^2 u+\lambda u = 0$ and its Applications. J. Math.~Phys. 31, 45--54 (1952).
    (See also P\'olya, G.: Remarks on the Foregoing Paper. J. Math.~Phys. 31, 55--57 (1952).)
  \bibitem{3} Payne, L. E., \& H. F. Weinberger: New Bounds for Solutions of Second
    Order Elliptic Partial Differential Equations. Pac.~J. of Math. 8, 551--573 (1958).
  \bibitem{4} Payne, L. E., \& H. F. Weinberger: Lower Bounds for Vibration Frequencies
    of Elastically Supported Membranes and Plates. J. Soc.~Indust.~Appl.~Math. 5, 171--182 (1957).
  \bibitem{5} Szeg\"o, G.: Inequalities for Certain Membranes of a Given Area.
    J. Rational Mech.\ Anal. 3, 343--356 (1954).
  \bibitem{6} Weinberger, H. F.: An Isoperimetric Inequality for the $N$-dimensional
    Free Membrane Problem. J. Rational Mech.\ Anal. 5, 633--636 (1956).
\end{thebibliography}
\end{document}